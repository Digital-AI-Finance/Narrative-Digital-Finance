\documentclass[8pt]{beamer}
\usetheme{Madrid}
\usepackage{graphicx}
\usepackage{booktabs}
\usepackage{adjustbox}
\usepackage{multicol}
\usepackage{amsmath}
\usepackage{amssymb}

% Color definitions matching template
\definecolor{mlblue}{RGB}{31, 119, 180}
\definecolor{mlorange}{RGB}{255, 127, 14}
\definecolor{mlgreen}{RGB}{44, 160, 44}
\definecolor{mlred}{RGB}{214, 39, 40}
\definecolor{mlpurple}{RGB}{148, 103, 189}
\definecolor{mlgray}{RGB}{127, 127, 127}

% Remove navigation symbols
\setbeamertemplate{navigation symbols}{}

% Clean itemize/enumerate
\setbeamertemplate{itemize items}[circle]
\setbeamertemplate{enumerate items}[default]

% Reduce margins for more content space
\setbeamersize{text margin left=5mm,text margin right=5mm}

% Title information
\title{Quantifying Narratives and Their Impact on Financial Markets}
\subtitle{Advanced NLP Methods for Systematic Trading Strategies}
\author{Prof. Dr. Joerg Osterrieder}
\institute{Based on: Bhargava et al. (2022) - State Street Associates}
\date{\today}

\begin{document}

% Title slide
\begin{frame}[t]
\titlepage
\end{frame}

% Table of contents
\begin{frame}[t]{Presentation Overview}
\tableofcontents
\vfill
\footnotesize
\textbf{Research Approach:} Systematic quantification of market narratives using NLP on 150,000+ digital sources. Rolling regression analysis establishes predictive relationships. Portfolio construction via narrative beta methodology. Out-of-sample validation demonstrates economic significance with IR = 1.26.
\end{frame}

% Section 1: Introduction and Motivation
\section{Introduction and Narrative Economics}

\begin{frame}[t]
\vfill
\centering
\begin{beamercolorbox}[sep=8pt,center]{title}
\usebeamerfont{title}\Large Introduction and Narrative Economics\par
\end{beamercolorbox}
\vfill
\end{frame}

\begin{frame}[t]{The Narrative Economics Framework}
\begin{columns}[T]
\begin{column}{0.38\textwidth}
\textbf{Shiller's Hypothesis:}
\begin{itemize}
\item Stories drive markets
\item Contagion effects
\item Measurable impact
\item Predictive power
\end{itemize}
\vspace{10pt}
\textbf{Our Contribution:}\\
Systematic quantification of narrative intensity
\end{column}
\begin{column}{0.58\textwidth}
\includegraphics[width=\textwidth]{narrative_intensity_vix.pdf}
\end{column}
\end{columns}
\vfill
\footnotesize
\textbf{Key Finding:} Market Crash narrative explains 34\% of SPY return variation. Correlation with VIX = 0.62, suggesting narratives capture additional information beyond traditional volatility measures. Real-time tracking enables dynamic portfolio allocation.
\end{frame}

% Section 2: Data and Methodology
\section{Data Architecture and NLP Pipeline}

\begin{frame}[t]
\vfill
\centering
\begin{beamercolorbox}[sep=8pt,center]{title}
\usebeamerfont{title}\Large Data Architecture and NLP Pipeline\par
\end{beamercolorbox}
\vfill
\end{frame}

\begin{frame}[t]{Data Processing Infrastructure}
\centering
\includegraphics[width=0.95\textwidth]{data_pipeline.pdf}
\vspace{5pt}
\small Key insight: Real-time narrative quantification from 150,000+ sources
\vfill
\footnotesize
\textbf{NLP Methods:} Transformer-based models for narrative identification. Sentiment scoring with reservoir adjustment for time-decay. Entity recognition links narratives to specific assets. Multilingual processing covers 15 languages. Cloud-based infrastructure processes 1M articles/day.
\end{frame}

\begin{frame}[t]{Narrative Intensity Measures}
\begin{columns}[T]
\begin{column}{0.25\textwidth}
\textbf{73 Narratives:}
\begin{itemize}
\item Market Crash
\item Gov Debt
\item COVID-19
\item Trade War
\item Inflation
\end{itemize}
\vspace{10pt}
\textbf{Key Metrics:}
\begin{itemize}
\item Intensity $I_{n,t}$
\item Negative $NI_{n,t}$
\item 7-day average
\end{itemize}
\end{column}
\begin{column}{0.75\textwidth}
\includegraphics[width=\textwidth]{r_squared_heatmap.pdf}
\end{column}
\end{columns}
\end{frame}

% Section 3: Mathematical Framework
\section{Mathematical Framework and Models}

\begin{frame}[t]
\vfill
\centering
\begin{beamercolorbox}[sep=8pt,center]{title}
\usebeamerfont{title}\Large Mathematical Framework and Models\par
\end{beamercolorbox}
\vfill
\end{frame}

\begin{frame}[t]{Core Regression Models}
\centering
\includegraphics[width=0.9\textwidth]{rolling_coefficients.pdf}
\begin{columns}[T]
\begin{column}{0.3\textwidth}
\small\textcolor{mlblue}{Blue: Coefficient}
\end{column}
\begin{column}{0.3\textwidth}
\small\textcolor{mlorange}{Orange: t-statistic}
\end{column}
\begin{column}{0.3\textwidth}
\small\textcolor{mlgreen}{Green: Significance}
\end{column}
\end{columns}
\vfill
\footnotesize
\textbf{Regression Specification:} $R_{t+1} = \alpha + \beta_1 \Delta NI_{n,t} + \beta_2 VIX_t + \beta_3 R_t + \epsilon_t$. HAC standard errors with Newey-West adjustment. 3-month rolling windows for parameter stability. Standardization via 60-day z-scores ensures comparability across narratives.
\end{frame}

% New slide with formula at bottom
\begin{frame}[t]{Portfolio Optimization with Narrative Constraints}
\textbf{Mean-Variance Framework with Narrative Exposure}

The optimal portfolio incorporates narrative sensitivities as additional constraints:

\begin{itemize}
\item Maximize risk-adjusted returns
\item Control narrative beta exposure
\item Dynamic rebalancing based on z-scores
\end{itemize}

\vfill
\begin{equation}
\max_w \left\{ w^T \mu - \frac{\lambda}{2} w^T \Sigma w + \gamma \sum_{j} \alpha_j E_j(w) \right\}
\end{equation}
\small
where $E_j(w) = \sum_i w_i \beta_{i,j}^{narrative}$ represents portfolio exposure to narrative $j$
\end{frame}

% Section 4: Empirical Results
\section{Core Empirical Results}

\begin{frame}[t]
\vfill
\centering
\begin{beamercolorbox}[sep=8pt,center]{title}
\usebeamerfont{title}\Large Core Empirical Results\par
\end{beamercolorbox}
\vfill
\end{frame}

\begin{frame}[t]{Dynamic Asset Allocation Performance}
\includegraphics[width=\textwidth]{allocation_performance.pdf}
\vspace{5pt}
\centering
\small Narrative-based strategy achieves Information Ratio = 1.26
\vfill
\footnotesize
\textbf{Strategy Rules:} Monitor Market Crash narrative z-score in real-time. Rotate from equity to bonds when z > 3 (extreme narrative intensity). Hold defensive position for 2 weeks minimum. Implementation lag: 2 days for execution. Annual return: 18.13\% vs 11.2\% benchmark.
\end{frame}

\begin{frame}[t]{3D Narrative Surface Analysis}
\centering
\includegraphics[width=0.85\textwidth]{3d_surface_plot.pdf}
\vfill
\footnotesize
\textbf{Surface Interpretation:} Expected returns decline with both narrative intensity and market volatility. Non-linear interaction effects visible at extreme values. Optimal allocation zone: low narrative intensity, moderate volatility. Model explains 42\% of return variation in 3D space.
\end{frame}

% New slide with technical text at bottom
\begin{frame}[t]{COVID-19 Case Study}
\textbf{Narrative Beta Portfolio Construction}

\begin{columns}[T]
\begin{column}{0.45\textwidth}
\textbf{Methodology:}
\begin{itemize}
\item Extract COVID narrative beta
\item Sort S\&P 500 constituents
\item Long/short quintiles
\item Monthly rebalancing
\end{itemize}
\end{column}
\begin{column}{0.45\textwidth}
\textbf{Performance:}
\begin{itemize}
\item Return: +120.74\%
\item Nov 2020 - Dec 2021
\item Vaccine pivot captured
\item Beat case-count strategy
\end{itemize}
\end{column}
\end{columns}

\vfill
\footnotesize
\textbf{Beta Calculation:} $\beta_{i,COVID} = \frac{Cov(R_i^{adj}, \Delta NI_{COVID})}{Var(\Delta NI_{COVID})}$ where $R_i^{adj}$ is market-adjusted return. Long portfolio: recovery plays (airlines, hotels). Short portfolio: lockdown beneficiaries (tech, e-commerce). Strategy captured narrative reversal on Pfizer announcement (Nov 9, 2020).
\end{frame}

% Section 5: Predictive Analysis
\section{Predictive Power and Model Comparison}

\begin{frame}[t]
\vfill
\centering
\begin{beamercolorbox}[sep=8pt,center]{title}
\usebeamerfont{title}\Large Predictive Power and Model Comparison\par
\end{beamercolorbox}
\vfill
\end{frame}

\begin{frame}[t]{Out-of-Sample Predictive Analysis}
\centering
\includegraphics[width=0.95\textwidth]{predictive_power_analysis.pdf}
\vspace{10pt}
\small Progressive improvement from adding narrative features
\vfill
\footnotesize
\textbf{Model Evolution:} Base model (lagged returns): $R^2$ = 2\%. Adding VIX: $R^2$ = 6\%. Adding narratives: $R^2$ = 12\%. Full model with interactions: $R^2$ = 18\%. COVID period shows enhanced predictive power of narratives during market stress.
\end{frame}

\begin{frame}[t]{Machine Learning Extensions}
\begin{columns}[T]
\begin{column}{0.38\textwidth}
\centering
\textbf{Model Performance}\\
\includegraphics[width=\textwidth]{ml_comparison.pdf}
\end{column}
\begin{column}{0.58\textwidth}
\centering
\textbf{COVID Beta Distribution}\\
\includegraphics[width=\textwidth]{covid_beta_distribution.pdf}
\end{column}
\end{columns}
\end{frame}

% Section 6: Portfolio Implementation
\section{Portfolio Implementation Strategies}

\begin{frame}[t]
\vfill
\centering
\begin{beamercolorbox}[sep=8pt,center]{title}
\usebeamerfont{title}\Large Portfolio Implementation Strategies\par
\end{beamercolorbox}
\vfill
\end{frame}

\begin{frame}[t]{Dynamic Allocation Framework}
\centering
\includegraphics[width=0.9\textwidth]{asset_allocation_pie.pdf}
\vspace{10pt}
\small Allocation shifts based on narrative intensity z-scores
\vfill
\footnotesize
\textbf{Implementation Details:} Daily narrative monitoring with 15-minute data updates. Position sizing via Kelly criterion with 25\% fraction. Risk limits: 15\% portfolio VaR, 20\% maximum drawdown. Transaction costs: 5bps equities, 2bps bonds. Rebalancing frequency optimized for cost-return tradeoff.
\end{frame}

% Multiple charts comparison
\begin{frame}[t]{Comparative Strategy Analysis}
\begin{columns}[T]
\begin{column}{0.48\textwidth}
\centering
\textbf{Network Effects}\\
\includegraphics[width=\textwidth]{narrative_network.pdf}
\end{column}
\begin{column}{0.48\textwidth}
\centering
\textbf{Factor Attribution}\\
\includegraphics[width=\textwidth]{ml_comparison.pdf}
\end{column}
\end{columns}
\vspace{5pt}
\begin{columns}[T]
\begin{column}{0.48\textwidth}
\centering
\small Network correlation > 0.6
\end{column}
\begin{column}{0.48\textwidth}
\centering
\small Sharpe Ratio comparison
\end{column}
\end{columns}
\end{frame}

% Key metrics with table
\begin{frame}[t]{Performance Metrics Summary}
\small
\begin{table}
\centering
\begin{tabular}{lcccc}
\toprule
Strategy & Annual Return & Volatility & Sharpe & Max DD \\
\midrule
SPY B\&H & 11.2\% & 18.5\% & 0.61 & -33.7\% \\
60/40 Portfolio & 8.7\% & 11.2\% & 0.78 & -18.2\% \\
VIX Timing & 13.1\% & 16.3\% & 0.80 & -22.1\% \\
\textcolor{mlgreen}{Narrative Strategy} & \textcolor{mlgreen}{18.1\%} & \textcolor{mlgreen}{14.4\%} & \textcolor{mlgreen}{1.26} & \textcolor{mlgreen}{-11.6\%} \\
ML Enhanced & 19.8\% & 15.7\% & 1.26 & -13.2\% \\
Combined & 20.3\% & 14.1\% & 1.44 & -10.3\% \\
\bottomrule
\end{tabular}
\end{table}
\vspace{10pt}
\includegraphics[width=0.8\textwidth]{allocation_performance.pdf}
\end{frame}

% Conclusion
\section{Conclusions and Future Research}

\begin{frame}[t]{Key Takeaways and Extensions}
\begin{columns}[T]
\begin{column}{0.4\textwidth}
\textbf{Contributions:}
\begin{itemize}
\item First systematic narrative quantification
\item 34\% explanatory power
\item IR = 1.26 strategy
\item Real-time implementation
\end{itemize}
\vspace{10pt}
\textbf{Future Research:}
\begin{itemize}
\item Transformer models
\item Cross-asset spillovers
\item High-frequency data
\end{itemize}
\end{column}
\begin{column}{0.6\textwidth}
\includegraphics[width=\textwidth]{predictive_power_analysis.pdf}
\end{column}
\end{columns}
\end{frame}

\begin{frame}[t]{Thank You}
\centering
\Large Questions and Discussion\\
\vspace{20pt}
\normalsize
\textbf{Contact:}\\
Prof. Dr. Joerg Osterrieder\\
\vspace{10pt}
\textbf{Paper:}\\
Bhargava et al. (2022)\\
SSA Research Paper\\
\vspace{10pt}
\textbf{Data \& Code:}\\
Available upon request
\end{frame}

\end{document}