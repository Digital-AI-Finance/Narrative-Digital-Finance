embeddings might introduce biases based on the pre-trained models used.

Use multiple embedding models and compare their performance in filtering research papers.

Comparison of clustering techniques?

accuracy of classification results?

Narratives gain or lose relevance over different market conditions

Use graph-based NLP methods



Financial Innovation
we will use financial innovation. Financial Innovation (FIN) is a Springer OA journal sponsored by Southwestern University of Finance and Economics. It provides a global academic forum for exchanging research findings across all fields in financial innovation in the era of electronic business that spans over several technological waves such as mobile computing, blockchain, and generative artificial intelligence (GenAI). It seeks to promote interactions among researchers, policy-makers, and practitioners, and to foster research ideas on financial innovation in the areas of new financial instruments as well as new financial technologies, markets and institutions.  FIN emphasizes emerging financial products, processes and services that are enabled by the introduction of disruptive technologies. FIN is peer-reviewed and publishes both high-quality academic (theoretical or empirical) and practical papers in the broad ranges of financial innovation. It has been indexed in SSCI, Scopus, Google Scholar, CNKI, CQVIP and so on.

Topic areas of interest include, but are not limited to, agentic financial workflow, asset pricing, behavioral finance, big data analytics in finance, computational financial intelligence, corporate finance, derivative pricing and hedging, disruptive financial models, extreme risks and insurance, financial economics, financial engineering, financial instruments, financial intermediation, financial market, financial risk management and analysis, GenAI-centric financial process automation, high frequency and algorithmic trading, household finance, human-AI collaboration in finance, innovative financial services, international finance, internet and mobile finance, legal and social issues of new finance, public finance and taxation, and other relevant topics. In particular, FIN welcomes research articles that study real-world impacts surrounding applications and management of blockchain, artificial intelligence, big data, Internet-of-Things, and other advanced computing such as GenAI adoption in banking and other finance services.

