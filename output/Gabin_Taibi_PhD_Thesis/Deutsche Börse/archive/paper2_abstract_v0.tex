This paper examines the impact and prevalence of High-Frequency Trading (HFT) in the Eurostoxx and DAX futures markets. We begin with a comprehensive literature review, detailing the evolution of HFT and the critical role of Field-Programmable Gate Array (FPGA) technology in enabling ultra-low latency trading. Our study then quantifies two key aspects of HFT activity: the proportion of market data generated by HFTs and their contribution to aggressive and passive traded volumes.

Using a unique nanosecond-granularity dataset from Deutsche Börse's datashop, we employ a novel classification methodology based on reaction times to trade-triggering events. Specifically, we categorize market participants into three groups: FPGA traders (reaction time < 1 microsecond), HFT (1-10 microseconds), and non-HFT (>10 microseconds). Our dataset comprises High-Precision Timestamp (HPT) files for all trade events, HPT All Files for all market events, and nanosecond-level orderbook updates from the DB A7 Analytics Platform for the most liquid instruments.
Our study quantifies two key aspects of algorithmic trading activity: the proportion of market data generated by each group and their contribution to aggressive and passive traded volumes. We calculate the participation rate and associated signal-to-noise ratio for FPGA traders, HFTs, and non-HFTs, providing a comprehensive view of their relative impact on market dynamics.

Findings reveal that [FPGA/HFT/non-HFT] traders account for ... of all market data messages in the FESX and FDAX futures markets, highlighting their varying influences on market microstructure. Furthermore, we find that [FPGA/HFT/non-HFT] are responsible for ... of aggressive orders and ... of passive orders, demonstrating their distinct roles in liquidity provision and consumption.

This research contributes to the growing body of literature on algorithmic trading by providing a detailed analysis of FPGA and HFT prevalence in European index futures markets, with a particular focus on nanosecond-level dynamics. Our results have important implications for market participants, regulators, and policymakers, offering insights into the technological arms race in trading infrastructure and its effects on market quality.