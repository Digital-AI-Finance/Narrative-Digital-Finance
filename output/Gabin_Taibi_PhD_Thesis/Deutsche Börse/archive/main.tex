\documentclass[preprint,12pt]{elsarticle}

\usepackage{amsmath}
\usepackage{graphicx}
\usepackage{hyperref}

\begin{document}

\begin{frontmatter}




\title{Nanoseconds traders: Analyzing the Rise and Role of High-Frequency Traders}



\author[1,2,3]{Gabin Taibi\corref{cor1}}
\ead{gabin.taibi@bfh.ch}

\author[3]{Stefan Schlamp\corref{cor1}}
\ead{stefan.schlamp@deutsche-boerse.com}

\author[1,2]{Joerg Osterrieder\corref{cor1}}
\ead{joerg.osterrieder@bfh.ch}

\address[1]{Department of Applied Data Science and Finance, Bern University of Applied Sciences, Bern, Switzerland}

\address[2]{Faculty of Behavioral Management and Social Sciences, University of Twente, Enschede, Netherlands}

\address[3]{Quantitative Analytics, Deutsche Börse, Eschborn, Germany}



\begin{abstract}
This paper examines the impact and prevalence of High-Frequency Trading (HFT) in the Eurostoxx and DAX futures markets. We begin with a comprehensive literature review, detailing the evolution of HFT and the critical role of Field-Programmable Gate Array (FPGA) technology in enabling ultra-low latency trading. Our study then quantifies two key aspects of HFT activity: the proportion of market data generated by HFTs and their contribution to aggressive and passive traded volumes.

Using a unique nanosecond-granularity dataset from Deutsche Börse's datashop, we employ a novel classification methodology based on reaction times to trade-triggering events. Specifically, we categorize market participants into three groups: FPGA traders (reaction time < 1 microsecond), HFT (1-10 microseconds), and non-HFT (>10 microseconds). Our dataset comprises High-Precision Timestamp (HPT) files for all trade events, HPT All Files for all market events, and nanosecond-level orderbook updates from the DB A7 Analytics Platform for the most liquid instruments.
Our study quantifies two key aspects of algorithmic trading activity: the proportion of market data generated by each group and their contribution to aggressive and passive traded volumes. We calculate the participation rate and associated signal-to-noise ratio for FPGA traders, HFTs, and non-HFTs, providing a comprehensive view of their relative impact on market dynamics.

Findings reveal that [FPGA/HFT/non-HFT] traders account for ... of all market data messages in the FESX and FDAX futures markets, highlighting their varying influences on market microstructure. Furthermore, we find that [FPGA/HFT/non-HFT] are responsible for ... of aggressive orders and ... of passive orders, demonstrating their distinct roles in liquidity provision and consumption.

This research contributes to the growing body of literature on algorithmic trading by providing a detailed analysis of FPGA and HFT prevalence in European index futures markets, with a particular focus on nanosecond-level dynamics. Our results have important implications for market participants, regulators, and policymakers, offering insights into the technological arms race in trading infrastructure and its effects on market quality.
\end{abstract}

\begin{keyword}
Financial Markets \sep High-Frequency Trading \sep Price Information \sep Information Share \sep Market Efficiency \sep Orderbook Liquidity \sep Market Micro structures \sep [specific models or methodologies]
\end{keyword}



\end{frontmatter}
\tableofcontents


\section{Introduction}
\label{sec:intro}

\subsection{The Limit Orderbook}
\begin{itemize}
    \item Electronic markets (not too long, removable)
    \item Order types (passive, aggressive), Prioritization (not too long, removable)
    \item T7 peculiarities
    \item Key instruments
\end{itemize}

\subsection{Technological Infrastructure}
\begin{itemize}
    \item Technological improvements (computational power, data lakes, collocation, FPGA)
    \item Deutsche Börse AG (subsidiaries, products, etc.)
    \item Deutsche Börse Systems (Eurex and T7)
\end{itemize}

\subsection{Market Participants}
\begin{itemize}
    \item Arbitragers, hedgers, speculators, market makers
    \item Market Makers vs. Market Takers
    \item Price formation (what is it? This is a well defined notion)
\end{itemize}

\subsection{Objectives and Structure of the Paper}
\begin{itemize}
    \item Research questions and objectives
    \item Overview of paper structure
    \item Summary of key contributions
\end{itemize}


\section{Literature Review}
\label{sec:litreview}


\subsection{Starting point and Technical improvements}

\subsection{Theory of efficient markets and critics}

\subsection{\textbf{Empirical studies on HFT}}

\subsection{Critics on HFT: market manipulation, latency arbitrage, leave markets when volatility is high so provoke flash crashes}
\begin{itemize}
\end{itemize}


\section{Data and Methodology}
\label{sec:datamethod}

\subsection{Description of Data}
\begin{itemize}
    \item Puiblic Data
    \item Periods (period of high volatility, period of low volatility)
    \item All trades (HPT files)
    \item All market events: trades, order add, modification, cancellation, etc. (HPT All files)
    \item Orderflow weighted microprice (A7 Algos)
\end{itemize}

\subsection{HFT, FPGA, Non-HFT Classification}
\begin{itemize}
    \item Post trade-events latencies
    \item Arbitrary threshold of 10 microseconds (for HFTs) and 1 microseconds (for FPGA)
    \item Metrics: percentage of HFTs participation, price formation impact (TODO once decided)
    \item Hypothesis
\end{itemize}



\subsection{Information Share Microprice}

\section{Empirical Results}
\label{sec:results}

\subsection{Description of Results}

\subsection{Participation of HFTs}

\subsection{HFTs impact on price formation}

\section{Robustness checks (XETRA and CME)}

\begin{itemize}
\end{itemize}


\section{Discussion}
\label{sec:discussion}


\subsection{Comparison with literature (if relevant)}

\subsection{Missclassification}
\subsection{Metrics used}

\subsection{Further Limitations}
\begin{itemize}
    \item Specific to DB data (only co-location, cable lengths, no micro waves)
    \item Arbitrary threshold for labeling as FPGA, HFTs and non-HFTs
    \item Possible presence of noise
    \item Impact on other layers of the orderbook?
\end{itemize}

\begin{itemize}
\end{itemize}


\section{Conclusion}
\label{sec:discussion}

\subsection{Summary of key findings}

\subsection{Future possible research}
\begin{itemize}
    \item Evolution of theses metrics? HFTs data as trading signal?
    \item Get non-public to correctly label HFTs/non-HFTs
    \item Do HFTs create or follow momentum?
    \item Comparison with US markets
\end{itemize}


\section*{Acknowledgements}
The author gratefully acknowledges:
\begin{itemize}
    \item The Deutsche Börse for hosting the PhD internship, particularly the Quantitative Analytics teams and \textbf{}Dr. Stefan Schlamp for supervision during the internship.
    \item The Swiss National Science Foundation (SNSF) for funding the PhD research project "Narrative Digital Finance: a tale of structural breaks, bubbles & market narratives" (grant number 213370).
    \item Bern University of Applied Sciences, the author's employer as a PhD student, and PhD supervisors Prof. Dr. Jörg Osterrieder and Prof. Dr. Branka Hadji Misheva.
    \item University of Twente, the host university awarding the PhD degree, and PhD promotor Prof. Dr. Martijn Mes.
    \item The COST Action 19130 FinAI for providing exposure to an important research network of researchers and industry members.
    \item The MSCA Digital Network on Digital Finance for offering access to leading industry partners and a vast network of researchers, professionals, and students.
\end{itemize}


\section*{References}
\bibliographystyle{elsarticle-num}
\bibliography{your_bibliography_file}

\end{document}

