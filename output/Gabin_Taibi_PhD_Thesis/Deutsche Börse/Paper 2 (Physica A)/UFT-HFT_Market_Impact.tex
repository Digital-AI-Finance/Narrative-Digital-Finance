\documentclass[preprint,12pt]{elsarticle}
\usepackage{amsmath}
\usepackage{graphicx}
\usepackage{hyperref}


\begin{document}

\begin{frontmatter}

\title{Nanoseconds traders: How Ultra-Fast and High-Frequency Traders Reshape Market Microstructure}

\author[1,2,3]{Gabin Taibi\corref{cor1}}
\ead{gabin.taibi@bfh.ch}

\author[3]{Stefan Schlamp\corref{cor1}}
\ead{stefan.schlamp@deutsche-boerse.com}

\author[1,2]{Joerg Osterrieder\corref{cor1}}
\ead{joerg.osterrieder@bfh.ch}

\address[1]{Department of Applied Data Science and Finance, Bern University of Applied Sciences, Bern, Switzerland}
\address[2]{Faculty of Behavioral Management and Social Sciences, University of Twente, Enschede, Netherlands}
\address[3]{Market Data and Services, Deutsche Börse, Eschborn, Germany}

\begin{abstract}
The advent of ultra-fast (UFT) and high-frequency trading (HFT) has fundamentally transformed financial markets, sparking intense debate about their impact on market quality and stability. This study provides a comprehensive empirical analysis of UFT and HFT effects on market microstructure, addressing critical questions about efficiency, liquidity, and volatility in modern electronic markets.

Utilizing high-precision timestamp data from Deutsche Börse's Eurex and Xetra platforms, we employ a novel classification methodology to categorize market participants based on their reaction latencies, distinguishing between FPGA-enabled ultra-fast traders, high-frequency traders, and conventional participants. Our analysis encompasses a diverse range of instruments, including futures, equities, and ETFs, focusing on the most liquid assets to capture the full spectrum of trading behavior.
We introduce a combination of eight metrics to quantify market quality, including effective spread, mid price, micro price, and trade price realized kernels for volatility estimation, alongside traditional measures such as volume, spread, depth and order imbalance ratio. A key contribution is our methodology for classifying trading activities as either "noise" or "signal," allowing for a nuanced examination of different trader categories' market impact.

% Our findings reveal a complex relationship between ultra-fast trading and market quality. While UFTs and HFTs contribute significantly to liquidity provision, we also observe instances of potential market manipulation through quote stuffing and latency arbitrage. The impact on volatility is multifaceted, with evidence suggesting both stabilizing and destabilizing effects under different market conditions.

% Contrary to some previous studies, we find that the "liquidity mirage" often attributed to high-speed traders is more nuanced, with FPGA-enabled participants demonstrating distinct behaviors from other HFTs. Our results also indicate that the market impact of ultra-fast trading varies across different asset classes and market conditions, challenging one-size-fits-all regulatory approaches.

This research contributes to the ongoing debate on market efficiency in the age of algorithmic trading, providing empirical evidence that both supports and refutes common criticisms of ultra-fast trading. Our findings have significant implications for market regulation, suggesting the need for more granular and adaptive oversight mechanisms to address the evolving landscape of electronic trading.
\end{abstract}

\begin{keyword}
Financial Markets \sep Ultra-Fast Trading \sep High-Frequency Trading \sep Price Information \sep Information Share \sep Market Efficiency \sep Orderbook Liquidity \sep Market Micro structures \sep [specific models or methodologies]
\end{keyword}

\end{frontmatter}


\newpage
\tableofcontents


\newpage
\section{Introduction}
\label{sec:intro}
Lorem ipsum

\section{Theoretical Background}
Lorem ipsum

\section{Literature Review}
Lorem ipsum

\section{Data and Methodology}
\label{sec:datamethod}
Lorem ipsum

\section{Empirical Results}
\label{sec:results}
Lorem ipsum

\section{Discussion}
\label{sec:discussion}
Lorem ipsum

\section{Conclusion}
\label{sec:conclusion}
Lorem ipsum


\section*{Acknowledgements}
\label{sec:acknowledgement}
The author gratefully acknowledges:
\begin{itemize}
    \item The Deutsche Börse for hosting the PhD internship, particularly the Quantitative Analytics teams and \textbf{}Dr. Stefan Schlamp for supervision during the internship.
    \item The Swiss National Science Foundation (SNSF) for funding the PhD research project "Narrative Digital Finance: a tale of structural breaks, bubbles \& market narratives" (grant number 213370).
    \item Bern University of Applied Sciences, the author's employer as a PhD student, and PhD supervisors Prof. Dr. Jörg Osterrieder and Prof. Dr. Branka Hadji Misheva.
    \item University of Twente, the host university awarding the PhD degree, and PhD promotor Prof. Dr. Martijn Mes.
    \item The COST Action 19130 FinAI for providing exposure to an important research network of researchers and industry members.
    \item The MSCA Digital Network on Digital Finance for offering access to leading industry partners and a vast network of researchers, professionals, and students.
\end{itemize}


\section*{References}
\bibliographystyle{elsarticle-num}
\bibliography{HFT_Activity_and_Market_Microstructures.bib}

\end{document}

