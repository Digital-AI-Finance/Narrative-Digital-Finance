\documentclass[12pt,a4paper]{article}

\usepackage[margin=1.8cm]{geometry}
\usepackage{setspace}
\setstretch{1.1}
\usepackage{titlesec}
\titleformat{\section}{\large\bfseries}{\thesection.}{0.5em}{}
\titleformat{\subsection}{\normalsize\bfseries}{\thesubsection}{0.5em}{}
\titleformat{\subsubsection}{\itshape}{\thesubsubsection}{0.5em}{}

\begin{document}

\begin{center}
    {\Large \textbf{PhD Thesis Outline}}\\[4pt]
    {\normalsize \textit{Modeling Narrative Dynamics for Volatility Regime Detection in Financial Markets}}\\[6pt]
\end{center}

\section*{Introduction: NLP and Textual Analysis for Narrative Modeling in Financial Markets}
\noindent
This chapter introduces the theoretical and methodological foundations of the thesis.  
It motivates the integration of natural language processing (NLP) with financial econometrics to quantify the influence of narratives on market dynamics.  
It defines key research objectives, outlines the data architecture (textual and high-frequency market data), and presents the conceptual link between narrative formation, information diffusion, and volatility regimes.

\section{Systematic Literature Review of Narratives for Financial Market Modeling}
\noindent
\textbf{Purpose:} Establish a clear conceptual and methodological baseline for the study of financial narratives.\\[2pt]
\textbf{Content:}
\begin{itemize}
    \item Define narratives in the context of finance and economics, distinguishing them from sentiment or topic modeling.
    \item Conduct an AI-enhanced systematic literature review (SLR) mapping how narrative concepts have evolved in financial research.
    \item Identify limitations of current studies and highlight the need for dynamic, quantitative narrative modeling.
\end{itemize}
\textbf{Note:} This chapter corresponds to the published or under-revision paper in \textit{Financial Innovation}.

\section{Key Financial Market Narratives}
\noindent
This chapter develops methods to detect and quantify key narratives across multiple text sources, combining supervised and unsupervised NLP approaches. Specifically, I apply 3 approaches to detect and track general market narratives over time: supervised embedding similarity-based, semi-supervised LLM-based, graph-based unsupervised online topic modeling.

\subsection{Market-Wide Narratives (News Headlines)}
\noindent
Using large-scale raw news headline datasets (RavenPack or LSEG or WallStreetJournal?). Compare to equity and indices prices and macro data.

\subsection{Macro Narratives (Central Bank Speeches)}
\noindent
Focuses on macroeconomic and policy-related narratives extracted from BIS and central bank speech transcripts.

- See 

Two possibilities for embedding-based narrative extraction:
- RAG-like embedding: split raw text in subtexts = 1-5 overlapping sentences OR paragraphs so we have a per-text narrative distribution
- Ask an LLM to summarise the speechs in one 5-10 sentences paragraphs with key information (state of the economy, forecasts, numbers, etc.) and embed each sentences OR the full summary

For LLM narrative extraction: give full text to the LLM with a custom secured prompt.

Compares macro narratives identified from dedicated policy texts with macroeconomic data (Interest Rates, CPI, PPI, unemployment, etc.) and narratives inferred from market headlines.

\subsection{Micro Narratives (Corporate Filings and Earnings Calls)}
\noindent
Analyzes company-specific narratives derived from SEC 10-K/10-Q filings and earnings call transcripts. Same embedding/LLM narrative extraction as previous section. Define company specific narratives in addition to broad narratives?
Assesses how firm-level narratives on risk, growth, or innovation connect to broader market sentiment: across 10K/10Q filings similarity = filings volatility, compare to news headlines for risk section management, and broad news narratives.

\subsection{Do Narratives Drive Markets?}
\noindent
Replicates and extends the Sadka et al. (2023) framework to evaluate whether macro and micro narratives explain market returns and volatility.  
Compares rolling \(R^2\) from headline-based versus source-specific narratives.  
\\[2pt]
\textbf{Expected outputs:} Two papers – (1) Quoniam collaboration paper on narrative detection; (2) Extension including macro/micro narrative comparison.

\noindent
Compare macro narrative predictive power vs. macro data.

\section{Market Microstructure and Volatility Modeling}
\noindent
This chapter builds the volatility modeling backbone from high-frequency trading data (Eurex \& Xetra).  
It benchmarks realized, implied, and rough volatility estimators and examines their relation to market microstructure.

\subsection{Market Participants and Microstructure}
\noindent
Analyzes Deutsche Börse nanosecond data to classify market participants (UFT, HFT, conventional) based on reaction time thresholds.  
Evaluates how their activity affects market quality metrics.

\subsection{Realized Volatility and Jump Estimation}
\noindent
Compares realized variance, realized kernel, MinRV, and MedRV estimators.  
Implements jump detection tests (BNS, ASJ) and consolidates results into a Python library (\texttt{realized-library}).

\subsection{Implied Volatility from Option Data}
\noindent
Estimates implied volatility and reconstructs volatility surfaces from tick-level option data.  
Serves as the bridge between market expectation and realized volatility.

\subsection{Volatility Second-Order Properties}
\noindent
Computes rolling Hurst exponent and volatility-of-volatility measures from both realized and implied volatilities.  
\\[2pt]
\textbf{Expected outputs:} Two papers – (1) HFT impact on market quality (submitted); (2) Volatility estimation benchmark (preprint).

\section{Narrative-Driven Volatility Structural Breaks}
\noindent
This chapter integrates textual and volatility features to detect, explain, and forecast volatility regime transitions.

\subsection{Volatility Structural Breaks}
\noindent
Applies structural break detection (Bai–Perron, CUSUM, Bayesian, Kernel) to realized, implied, and rough volatility measures.  
Analyzes narrative intensity patterns around breakpoints.

\subsection{Do Narratives Drive Volatility Dynamics?}
\noindent
Extends the Sadka framework to volatility metrics—examining whether narrative shifts precede changes in volatility, roughness, or vol-of-vol.

\subsection{Narrative Causality}
\noindent
Uses causal econometrics (Granger, Toda–Yamamoto, transfer entropy) and machine learning approaches (Double ML, Causal Forests) to identify which narratives causally drive volatility components.

\subsection{Volatility Forecasting}
\noindent
Develops narrative-enhanced forecasting models (HAR-RV, XGBoost, LSTM, Temporal Fusion Transformers).  
Tests out-of-sample predictive power and explores volatility factor construction.  
\\[2pt]
\textbf{Expected outputs:} Two papers – (1) “Do Narratives Drive Volatility Dynamics?” (preprint) and (2) “Narratives as Causal Drivers of Volatility Regimes” (target: \textit{JFQA}).

\section*{Conclusion: Synthesizing Insights on Narratives and Volatility}
\noindent
Summarizes the thesis contributions linking narrative detection, volatility structure, and regime dynamics.  
Emphasizes the interdisciplinary integration of NLP, econometrics, and high-frequency finance.  
Outlines implications for volatility modeling, risk management, and future research on cross-asset narrative contagion.

\end{document}
