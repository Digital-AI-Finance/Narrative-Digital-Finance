\subsection{Central Bank Communication}

The academic study of central bank communication has grown substantially since the Federal Reserve's shift toward greater transparency in the 1990s. \citet{blinder2008central} provides a comprehensive overview of how central bank communication has evolved and its impact on financial markets. The authors argue that communication has become an increasingly important policy tool, complementing traditional interest rate decisions.

\citet{woodford2005central} develops a theoretical framework for understanding how central bank communication affects expectations and economic outcomes. The paper emphasizes the importance of credibility and consistency in central bank messaging, arguing that effective communication can enhance the transmission of monetary policy.

The empirical literature has documented significant market reactions to central bank communications. \citet{gurkaynak2005sensitivity} shows that Federal Reserve announcements affect asset prices through both actions (rate decisions) and signals (communication about future policy). This finding has been replicated across multiple central banks and time periods.

\subsection{Speech-Based Analysis}

A growing literature focuses specifically on central bank speeches as opposed to formal policy statements. \citet{hansen2016shocking} use topic modeling to analyze Federal Reserve communications, finding that speeches contain valuable information about policy intentions beyond what is captured in formal statements.

\citet{apel2014foreign} analyze Swedish Riksbank speeches and find that hawkish communication is associated with higher interest rate expectations. Their methodology for classifying speech sentiment has been widely adopted in subsequent research.

More recently, \citet{shapiro2022measuring} develops a comprehensive news sentiment index for monetary policy, incorporating both formal communications and media coverage. The index shows significant predictive power for future policy decisions.

The latest wave of research has increasingly adopted deep learning methods for sentiment extraction. \citet{hilscher2024information} use FinBERT models to analyze Fed and ECB communication, finding evidence of cross-Atlantic sentiment spillovers. \citet{picault2023unveiling} develop novel deep learning indices for Central and Eastern European central banks, demonstrating that context-aware models outperform dictionary-based approaches. \citet{araujo2023central} construct a Central Bankers' Sentiment Index from 6,514 speeches across eight central banks, showing that Fed sentiment has the largest and most persistent international spillover effects. \citet{jansen2024ecb} examine how ECB communication sentiment relates to the economic environment and financial markets, finding that media coverage accurately transmits the sentiment of official communications.

\subsection{Macroeconomic Factor Models}

Our use of PCA to construct macroeconomic indices follows a rich tradition in empirical macroeconomics. \citet{stock2002forecasting} demonstrate that factor models can improve macroeconomic forecasting by efficiently summarizing information from large datasets.

\citet{ludvigson2009macro} extend this approach to financial applications, showing that macroeconomic factors have significant explanatory power for asset returns. Their work establishes the theoretical foundation for using PCA-based indices in financial analysis.

The use of rolling standardization to capture time-varying dynamics follows \citet{diebold2006measuring} who emphasize the importance of allowing for parameter instability in macroeconomic models.

\subsection{Structural Break Detection}

The detection of structural breaks in economic time series has a long history dating to \citet{chow1960tests}. More recent advances include the PELT algorithm developed by \citet{killick2012optimal}, which provides an efficient method for detecting multiple change points in time series data.

\citet{hamilton1989new} develops the influential Markov switching framework for modeling regime changes in economic data. While our approach differs by not imposing a specific parametric structure, we share the goal of identifying distinct economic regimes.

Applications to monetary policy include \citet{sims2006were} who document significant instability in the relationship between monetary policy and macroeconomic outcomes over time.

\subsection{Sentiment Analysis in Finance}

The application of natural language processing to finance has grown rapidly in recent years. \citet{loughran2011liability} develop a finance-specific sentiment dictionary that has become a standard tool in the field. Their key insight is that general-purpose sentiment dictionaries may not capture finance-specific meanings.

\citet{tetlock2007giving} pioneered the use of media sentiment to predict stock returns, showing that negative sentiment forecasts lower prices. This work has spawned a large literature on the predictive power of textual data for financial outcomes.

\citet{baker2016measuring} construct an economic policy uncertainty index based on newspaper coverage, demonstrating the value of textual data for measuring latent economic concepts.

\subsection{Narrative Economics}

Our work also connects to the emerging field of narrative economics championed by \citet{shiller2017narrative}. Shiller argues that economic narratives---the stories people tell about the economy---can have causal effects on economic outcomes by shaping expectations and behavior.

\citet{andre2022subjective} provide experimental evidence that narratives affect economic beliefs and decisions. Their findings suggest that central bank communication may influence the economy not just through information transmission but also through the stories it tells.

\subsection{Contribution to the Literature}

Our paper contributes to this literature in several ways. First, we provide a comprehensive framework for analyzing the relationship between central bank communication and macroeconomic conditions that combines multiple methodological approaches. Second, we document the striking finding that speech sentiment has essentially zero correlation with contemporaneous macroeconomic indices, contributing to debates about the effectiveness of central bank communication. Third, we provide evidence on the time-series properties of speech sentiment, including strong mean reversion, that has implications for understanding how central banks manage their communication over time.
