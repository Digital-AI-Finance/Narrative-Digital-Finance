\subsection{Main Finding: Near-Zero Correlation}

Our central finding is that the correlation between central bank speech sentiment and macroeconomic indices is essentially zero. Table \ref{tab:correlations} reports the correlation matrix for first-differenced series.

\begin{table}[H]
\centering
\caption{Correlation Matrix (First Differences)}
\label{tab:correlations}
\begin{tabular}{lrrrr}
\toprule
& Macro Index & Inflation Index & Hawkish & Dovish \\
\midrule
Macro Index & 1.000 & & & \\
Inflation Index & 0.042 & 1.000 & & \\
Hawkish & \textbf{0.005} & 0.003 & 1.000 & \\
Dovish & 0.005 & -0.026 & 0.098 & 1.000 \\
\bottomrule
\end{tabular}
\end{table}

The correlation between the Macro Index and Hawkish sentiment is $r = 0.005$, statistically indistinguishable from zero. Similarly, the correlation between the Inflation Index and Hawkish sentiment is $r = 0.003$. These results indicate that changes in central bank speech sentiment do not systematically coincide with changes in macroeconomic conditions.

\subsection{Autocorrelation Analysis}

Table \ref{tab:autocorr} reports the first-order autocorrelations for each series.

\begin{table}[H]
\centering
\caption{First-Order Autocorrelations}
\label{tab:autocorr}
\begin{tabular}{lr}
\toprule
Variable & $\rho_1$ \\
\midrule
Macro Index (PC1) & 0.052 \\
Inflation Index (PC2) & 0.288 \\
Hawkish Sentiment & -0.430 \\
Dovish Sentiment & -0.380 \\
\bottomrule
\end{tabular}
\end{table}

The macroeconomic indices show low to moderate persistence, with the Inflation Index more persistent than the Macro Index. In contrast, speech sentiment shows strong \textit{negative} autocorrelation, indicating rapid mean reversion. A hawkish month is typically followed by a less hawkish month, and vice versa.

\subsection{Structural Break Analysis}

The PELT algorithm identifies significant structural breaks in both macroeconomic indices.

\subsubsection{Macro Strength Index Breakpoints}

Eight breakpoints are detected in the Macro Strength Index, dividing the sample into nine distinct regimes:
\begin{enumerate}
    \item January 2001: Dot-com bust beginning
    \item December 2003: Recovery consolidation
    \item April 2007: Pre-crisis peak
    \item March 2010: Post-GFC recovery
    \item November 2016: Post-election regime
    \item May 2019: Late-cycle slowdown
    \item June 2021: COVID recovery
    \item July 2023: Tightening cycle peak
\end{enumerate}

\subsubsection{Inflation Index Breakpoints}

Twelve breakpoints are detected in the Inflation Index, reflecting greater volatility in price dynamics over the sample period. Key breaks include the commodity boom (2007-2008), the disinflation following the Global Financial Crisis (2009-2010), and the post-pandemic inflation surge (2021-2022).

\subsection{Rolling Regression Results}

Figure \ref{fig:rolling_betas} presents the rolling 36-month betas for Hawkish and Dovish sentiment on the Macro Index. The betas fluctuate around zero throughout the sample period, with no persistent positive or negative relationship.

Figure \ref{fig:rolling_r2} shows the corresponding rolling $R^2$ values. The $R^2$ is consistently below 0.05, indicating that speech sentiment explains less than 5\% of the variation in macroeconomic index changes at any point in time. The average $R^2$ over the sample period is approximately 0.01.

\subsection{Robustness Checks}

\subsubsection{Alternative Lag Structures}

We examine whether speech sentiment leads or lags macroeconomic conditions by computing cross-correlations at various lags. The results show no significant correlation at any lag from -12 to +12 months.

\subsubsection{Subperiod Analysis}

We divide the sample into three subperiods: pre-GFC (1996-2007), GFC and aftermath (2008-2015), and recent period (2016-2025). The near-zero correlation finding holds in all three subperiods.

\subsubsection{Alternative Standardization}

We test alternative standardization methods including fixed-window standardization and no standardization. The results are robust to these alternatives.

\subsection{Implications}

Our findings have several important implications:

\begin{enumerate}
    \item \textbf{Weak information content}: Central bank speeches do not provide significant real-time information about current macroeconomic conditions.

    \item \textbf{Forward-looking communication}: The low contemporaneous correlation may reflect that central bankers speak primarily about future expectations rather than current conditions.

    \item \textbf{Mean-reverting communication}: The strong negative autocorrelation suggests that central banks actively manage the ``balance'' of their communication, avoiding prolonged hawkish or dovish stretches.

    \item \textbf{Regime independence}: Speech sentiment does not systematically anticipate or respond to the structural breaks detected in macroeconomic indices.
\end{enumerate}
