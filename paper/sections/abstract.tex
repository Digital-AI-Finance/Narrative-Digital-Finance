This paper investigates the relationship between Federal Reserve speech sentiment and macroeconomic conditions using a comprehensive analytical framework combining Principal Component Analysis (PCA), structural break detection, and rolling regression analysis. Analyzing 2,421 US Fed speeches from 1996 to 2025 alongside six FRED macroeconomic indicators, we construct orthogonal indices capturing macroeconomic strength and inflation dynamics. Our key finding reveals a \textbf{near-zero correlation} ($r = 0.005$) between central bank speech sentiment (hawkish/dovish) and macroeconomic indices, challenging conventional assumptions about the effectiveness of narrative-driven policy transmission. The Pruned Exact Linear Time (PELT) algorithm identifies 8 macro regime breakpoints and 12 inflation regime breakpoints over the sample period, demonstrating significant structural instability. Rolling regression analysis shows consistently low explanatory power ($R^2 < 0.01$) of sentiment for macroeconomic changes, while strong negative autocorrelation ($\rho = -0.43$) in hawkish sentiment suggests rapid mean reversion in central bank communication. These results have important implications for understanding central bank communication effectiveness, forward guidance credibility, and the disconnect between economic narratives and underlying macroeconomic conditions.
