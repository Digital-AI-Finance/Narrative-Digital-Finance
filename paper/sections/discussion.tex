\subsection{Interpretation of Main Results}

The near-zero correlation between central bank speech sentiment and macroeconomic indices is a striking finding that invites multiple interpretations.

\subsubsection{Forward-Looking Communication Hypothesis}

One interpretation is that central bank communication is primarily forward-looking. If Fed officials speak about expected future conditions rather than current conditions, contemporaneous correlations would be weak even if communication is highly informative. This interpretation is consistent with the forward guidance literature emphasizing the importance of managing expectations about future policy.

Under this hypothesis, we would expect speech sentiment to lead macroeconomic changes. However, our cross-correlation analysis shows no significant lead relationship, casting doubt on this interpretation or suggesting very long lead times beyond our 12-month window.

\subsubsection{Strategic Communication Hypothesis}

An alternative interpretation is that central bank communication serves strategic purposes beyond information transmission. Fed officials may deliberately avoid commenting on current conditions to maintain policy flexibility, or they may use speeches to manage market expectations independently of underlying fundamentals.

The strong negative autocorrelation in sentiment supports this view. The pattern suggests active management of communication tone, with hawkish messages followed by dovish messages as if to maintain balance or avoid appearing biased in one direction.

\subsubsection{Narrative-Reality Disconnect}

A third interpretation aligns with the ``narrative economics'' perspective of \citet{shiller2017narrative}. Central bank speeches may reflect broader economic narratives that circulate in society rather than precise assessments of current conditions. These narratives have their own dynamics that may be loosely coupled with macroeconomic reality.

\subsection{Implications for Monetary Policy}

\subsubsection{Forward Guidance Effectiveness}

Our findings raise questions about the effectiveness of forward guidance as a policy tool. If the information content of speeches about current conditions is low, the value of speeches as signals about future policy may also be limited. Market participants may need to look beyond speech sentiment to other indicators for reliable policy signals.

\subsubsection{Communication Strategy}

The mean-reversion in speech sentiment suggests that the Federal Reserve maintains a deliberate communication strategy aimed at balancing hawkish and dovish messages over time. This pattern may reflect institutional norms, concern about appearing biased, or active management of market expectations.

Central banks may benefit from greater consistency in their messaging or from more explicit acknowledgment of current macroeconomic conditions in their communications.

\subsection{Implications for Financial Markets}

\subsubsection{Trading on Speech Sentiment}

Our results suggest that trading strategies based on central bank speech sentiment may have limited profitability if the underlying assumption is that sentiment reflects current macroeconomic conditions. The near-zero correlation means that sentiment provides little edge for predicting macroeconomic-driven price movements.

However, speech sentiment may still be valuable for other purposes, such as predicting specific policy decisions or capturing market reactions to communication events.

\subsubsection{Information Extraction}

Financial market participants may need more sophisticated methods for extracting information from central bank communications. Rather than simple sentiment classifications, topic-specific analysis or extraction of forward-looking statements may provide more value.

\subsection{Limitations}

\subsubsection{Sentiment Classification}

Our analysis relies on automated sentiment classification, which may not fully capture the nuances of central bank communication. Speeches may contain mixed signals, conditional statements, or context-dependent meanings that simple classifications miss.

\subsubsection{Sample Period}

While our sample spans nearly three decades, certain results may be specific to the particular monetary policy regimes in this period. The findings may not generalize to other central banks or historical periods with different communication practices.

\subsubsection{Causality}

Our correlation-based analysis cannot establish causal relationships. Low correlation could reflect true independence, bidirectional causality, or confounding factors that obscure the underlying relationship.

\subsection{Future Research}

Several directions for future research emerge from our findings:

\begin{enumerate}
    \item \textbf{Topic-specific analysis}: Examining whether specific topics (inflation, employment, financial stability) show stronger correlations with relevant macroeconomic indicators.

    \item \textbf{Cross-country comparison}: Investigating whether the narrative-reality disconnect characterizes other central banks or is specific to the Federal Reserve.

    \item \textbf{Historical analysis}: Extending the analysis to earlier periods to understand how communication practices have evolved.

    \item \textbf{Market reactions}: Analyzing whether markets react differently to speeches depending on alignment with macroeconomic conditions.
\end{enumerate}
