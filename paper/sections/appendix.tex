\subsection{Figures}

\begin{figure}[H]
\centering
\includegraphics[width=0.9\textwidth]{figures/05_speech_count_distribution.pdf}
\caption{Monthly Distribution of Federal Reserve Speeches by Sentiment Category (1996-2025). Hawkish speeches (red) suggest tighter monetary policy, dovish speeches (blue) suggest looser policy, and neutral speeches (gray) show no clear direction.}
\label{fig:speech_dist}
\end{figure}

\begin{figure}[H]
\centering
\includegraphics[width=0.9\textwidth]{figures/07_rolling_betas_macro.pdf}
\caption{Rolling 36-Month Betas: Speech Sentiment on Macro Strength Index. The betas for both hawkish and dovish sentiment fluctuate around zero throughout the sample period, indicating no persistent relationship between speech sentiment and macroeconomic conditions.}
\label{fig:rolling_betas}
\end{figure}

\begin{figure}[H]
\centering
\includegraphics[width=0.9\textwidth]{figures/08_rolling_r2_macro.pdf}
\caption{Rolling 36-Month $R^2$: Speech Sentiment Explaining Macro Strength Index. The $R^2$ remains consistently below 0.05 throughout the sample, indicating that speech sentiment explains less than 5\% of the variation in macroeconomic index changes at any point in time.}
\label{fig:rolling_r2}
\end{figure}

\subsection{Additional PCA Results}

Table \ref{tab:pca_full} presents the full PCA loadings for all six principal components.

\begin{table}[H]
\centering
\caption{Complete PCA Loadings and Variance Explained}
\label{tab:pca_full}
\small
\begin{tabular}{lrrrrrr|r}
\toprule
& PC1 & PC2 & PC3 & PC4 & PC5 & PC6 & Var. Exp. \\
\midrule
Fed Funds Rate & 61.3 & -31.2 & 71.6 & 10.5 & 2.1 & 4.6 & --- \\
CPI & 11.3 & 35.0 & -5.7 & 35.7 & 42.1 & 74.6 & --- \\
PPI & 33.7 & 68.9 & -3.1 & -47.9 & -40.1 & 16.0 & --- \\
GDP & 24.7 & -30.9 & 13.9 & 36.9 & 51.9 & -54.0 & --- \\
Unemployment & -40.9 & 15.7 & 51.9 & 36.2 & -50.9 & -9.9 & --- \\
Nonfarm Payrolls & 48.0 & -2.2 & -47.2 & 46.0 & -57.9 & 0.9 & --- \\
\midrule
Var. Explained (\%) & 48.6 & 23.2 & 15.4 & 6.2 & 4.0 & 2.7 & 100.0 \\
Cumulative (\%) & 48.6 & 71.8 & 87.1 & 93.4 & 97.3 & 100.0 & --- \\
\bottomrule
\end{tabular}
\end{table}

\subsection{Breakpoint Details}

\subsubsection{Macro Strength Index Regimes}

Table \ref{tab:macro_regimes} characterizes each regime identified by the PELT algorithm in the Macro Strength Index.

\begin{table}[H]
\centering
\caption{Macro Strength Index Regimes}
\label{tab:macro_regimes}
\begin{tabular}{llrrrr}
\toprule
Regime & Period & Duration & Mean & Std Dev & Label \\
\midrule
1 & 1997-01 to 2001-01 & 49 mo & 1.24 & 1.15 & Dot-com Boom \\
2 & 2001-02 to 2003-12 & 35 mo & -1.42 & 0.89 & Recession \\
3 & 2004-01 to 2007-04 & 40 mo & 0.87 & 0.78 & Recovery \\
4 & 2007-05 to 2010-03 & 35 mo & -1.89 & 1.56 & GFC Crisis \\
5 & 2010-04 to 2016-11 & 80 mo & 0.23 & 0.92 & Slow Recovery \\
6 & 2016-12 to 2019-05 & 30 mo & 1.45 & 0.67 & Late Expansion \\
7 & 2019-06 to 2021-06 & 25 mo & -0.78 & 2.34 & COVID Impact \\
8 & 2021-07 to 2023-07 & 25 mo & 1.12 & 0.91 & Recovery \\
9 & 2023-08 to 2025-05 & 22 mo & -0.34 & 0.76 & Normalization \\
\bottomrule
\end{tabular}
\end{table}

\subsubsection{Inflation Index Regimes}

Table \ref{tab:inflation_regimes} characterizes each regime identified by the PELT algorithm in the Inflation Index.

\begin{table}[H]
\centering
\caption{Inflation Index Regimes}
\label{tab:inflation_regimes}
\small
\begin{tabular}{llrrl}
\toprule
Regime & Period & Duration & Mean & Label \\
\midrule
1 & 1997-01 to 1999-04 & 28 mo & 0.45 & Moderate Inflation \\
2 & 1999-05 to 2001-05 & 25 mo & -0.82 & Low Inflation \\
3 & 2001-06 to 2002-03 & 10 mo & 0.67 & Rising Prices \\
4 & 2002-04 to 2004-04 & 25 mo & -0.34 & Disinflation \\
5 & 2004-05 to 2007-03 & 35 mo & 1.23 & Commodity Boom \\
6 & 2007-04 to 2008-06 & 15 mo & 2.45 & Oil Price Spike \\
7 & 2008-07 to 2014-09 & 75 mo & -0.56 & Post-GFC Disinflation \\
8 & 2014-10 to 2016-05 & 20 mo & -1.12 & Oil Price Collapse \\
9 & 2016-06 to 2018-06 & 25 mo & 0.78 & Gradual Normalization \\
10 & 2018-07 to 2020-07 & 25 mo & 0.12 & Pre-COVID Stability \\
11 & 2020-08 to 2022-08 & 25 mo & 1.89 & Pandemic Inflation \\
12 & 2022-09 to 2023-11 & 15 mo & 0.56 & Inflation Moderation \\
13 & 2023-12 to 2025-05 & 18 mo & -0.23 & Rate Cut Anticipation \\
\bottomrule
\end{tabular}
\end{table}

The Inflation Index exhibits more regime changes (12 breakpoints) than the Macro Strength Index (8 breakpoints), reflecting greater volatility in price dynamics over the sample period.

\subsection{Rolling Regression Details}

\subsubsection{Window Sensitivity}

Table \ref{tab:window_sensitivity} shows how the average correlation varies with rolling window size.

\begin{table}[H]
\centering
\caption{Window Size Sensitivity Analysis}
\label{tab:window_sensitivity}
\begin{tabular}{lrrrr}
\toprule
Window (months) & Mean $\beta^H$ & Mean $\beta^D$ & Mean $R^2_H$ & Mean $R^2_D$ \\
\midrule
24 & 0.003 & 0.002 & 0.008 & 0.006 \\
36 & 0.004 & 0.003 & 0.010 & 0.008 \\
48 & 0.003 & 0.002 & 0.009 & 0.007 \\
60 & 0.003 & 0.002 & 0.008 & 0.007 \\
\bottomrule
\end{tabular}
\end{table}

Results are robust to window size choice, with consistently near-zero betas and low $R^2$ values.

\subsection{Subperiod Results}

Table \ref{tab:subperiod} reports correlations for three subperiods.

\begin{table}[H]
\centering
\caption{Subperiod Correlation Analysis}
\label{tab:subperiod}
\begin{tabular}{lrrr}
\toprule
Period & Macro-Hawkish & Inflation-Hawkish & Obs \\
\midrule
1996-2007 & 0.008 & 0.005 & 144 \\
2008-2015 & 0.003 & -0.012 & 96 \\
2016-2025 & 0.006 & 0.008 & 100 \\
\midrule
Full Sample & 0.005 & 0.003 & 340 \\
\bottomrule
\end{tabular}
\end{table}

The near-zero correlation finding is consistent across all subperiods.
