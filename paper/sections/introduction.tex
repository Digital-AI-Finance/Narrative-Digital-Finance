Central bank communication has become an increasingly important tool of monetary policy over the past three decades. Since the Federal Reserve's shift toward greater transparency in the 1990s, speeches, press conferences, and written statements have joined interest rate decisions as key channels through which monetary policy influences financial markets and the broader economy. The effectiveness of this communication, however, depends critically on the relationship between what central bankers say and the underlying economic conditions they face.

This paper investigates a fundamental question: \textit{Do central bank speeches reflect contemporaneous macroeconomic conditions?} If central bank communication is primarily reactive to current economic data, we would expect strong correlations between speech sentiment and macroeconomic indices. Alternatively, if central banks are primarily forward-looking or if their communication serves other strategic purposes, the contemporaneous relationship may be weak or absent.

Understanding this relationship has important implications for several areas of monetary economics. First, it informs debates about the effectiveness of forward guidance as a policy tool. If central bank speeches do not systematically reflect current conditions, their value as signals about future policy may be limited. Second, it has implications for financial market participants who parse central bank communications for trading signals. Third, it contributes to the growing literature on narrative economics and the role of stories in shaping economic expectations.

We make three main contributions to the literature. First, we develop a comprehensive framework for analyzing the relationship between central bank communication and macroeconomic conditions, combining Principal Component Analysis (PCA) for dimensionality reduction, the Pruned Exact Linear Time (PELT) algorithm for structural break detection, and rolling regression analysis for time-varying relationships. Second, we provide novel empirical evidence on the correlation between speech sentiment and macroeconomic indices using a large dataset of 2,421 Federal Reserve speeches spanning nearly three decades (1996-2025). Third, we document a striking \textit{narrative-reality disconnect}: the near-zero correlation between what central bankers say and current macroeconomic conditions.

Our analysis proceeds as follows. We first construct macroeconomic indices using PCA on six key indicators: the Federal Funds Rate, Consumer Price Index (CPI), Producer Price Index (PPI), GDP, unemployment rate, and nonfarm payrolls. The first principal component (PC1) captures overall macroeconomic strength, loading positively on the policy rate and employment while loading negatively on unemployment. The second principal component (PC2) captures inflation dynamics, loading heavily on price indices while negatively loading on the policy rate. These two components explain 72\% of the variation in the underlying indicators.

We then aggregate speech sentiment from the BIS central bank speeches database, classifying each speech as hawkish, dovish, or neutral based on natural language processing techniques. Monthly sentiment scores are computed and standardized using a rolling 12-month window to capture deviations from recent trends.

Our main finding is that the correlation between speech sentiment and macroeconomic indices is essentially zero ($r = 0.005$ for the macro strength index, $r = 0.003$ for the inflation index). This result is robust to various specifications including different lag structures, alternative standardization methods, and subperiod analysis. Rolling regressions confirm that this finding persists throughout the sample period, with $R^2$ values consistently below 1\%.

The PELT algorithm identifies significant structural breaks in both macroeconomic indices, with 8 breakpoints in the macro strength index and 12 in the inflation index. Key breaks correspond to well-known economic events including the dot-com bust (2001), the housing bubble peak (2007), the COVID-19 pandemic (2020), and the post-pandemic inflation surge (2022). Notably, speech sentiment does not systematically anticipate or respond to these regime changes.

The strong negative autocorrelation in hawkish sentiment ($\rho = -0.43$) suggests that central bank communication exhibits rapid mean reversion, with hawkish months typically followed by more dovish months and vice versa. This pattern may reflect deliberate communication strategies aimed at maintaining balance or may simply capture the volatile nature of economic commentary.

The remainder of this paper is organized as follows. Section 2 reviews the relevant literature on central bank communication, sentiment analysis, and macroeconomic indices. Section 3 describes our methodology including the PCA framework, PELT algorithm, and rolling regression approach. Section 4 presents the data sources and summary statistics. Section 5 reports our main empirical results. Section 6 discusses the implications of our findings for monetary policy and financial markets. Section 7 concludes.
