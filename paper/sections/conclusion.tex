This paper investigates the relationship between Federal Reserve speech sentiment and macroeconomic conditions using a comprehensive analytical framework. Our main finding is a \textit{narrative-reality disconnect}: the correlation between speech sentiment and macroeconomic indices is essentially zero ($r = 0.005$).

We construct macroeconomic indices using Principal Component Analysis on six FRED indicators, with the first two components explaining over 70\% of variance and capturing macro strength and inflation dynamics respectively. The PELT algorithm identifies 8 structural breaks in the Macro Strength Index and 12 in the Inflation Index, corresponding to well-known economic events including the dot-com bust, the Global Financial Crisis, and the post-pandemic inflation surge.

Analysis of 2,421 Federal Reserve speeches reveals that sentiment shows strong negative autocorrelation ($\rho = -0.43$), indicating rapid mean reversion in central bank communication. Rolling regressions confirm that the explanatory power of sentiment for macroeconomic changes is consistently below 1\% throughout the sample period.

These findings have important implications for monetary policy and financial markets. First, they suggest that central bank speeches may not primarily reflect current macroeconomic conditions, whether because communication is forward-looking, strategic, or loosely coupled with economic fundamentals. Second, the mean-reversion pattern suggests active management of communication tone by the Federal Reserve. Third, market participants relying on speech sentiment as a proxy for current conditions may need to revise their approach.

Our analysis contributes to the growing literature on central bank communication, narrative economics, and the effectiveness of forward guidance. While the findings do not diminish the importance of central bank communication for financial markets and policy transmission, they highlight the complexity of the information conveyed and the challenges of extracting macroeconomic signals from speeches.

Future research should explore whether topic-specific analysis reveals stronger relationships, whether similar patterns characterize other central banks, and how market reactions to speeches depend on alignment with underlying economic conditions. Understanding the narrative-reality disconnect in central bank communication remains an important challenge for monetary economics.
